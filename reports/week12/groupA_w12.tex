\documentclass[11pt, a4paper]{article}

\begin{document}
\title{}
\author{Groep A}
\date{14 mei 2014}
\maketitle

\section{Aanwezigheden}
Iedereen was aanwezig.
\section{Status}
We zijn de verplichte functionaliteit aan het afwerken. Via Github issues houden we bij wat nog moet gebeuren.
\subsection{Stijn}
Veel meer data gecrawld. Nieuwe velden (zoals spelerpositie) toegevoegd.
\subsection{Kristof}
Bouwde notificatiesysteem. Bouwde user group systeem.
\subsection{Jakob}
Verbeterde en breidde betsysteem uit.
\subsection{Ruben}
..
\subsection{Tom}
Bugfixes

\section{Besproken onderwerpen}
De vergaderingen in persoon hebben grotendeels plaats gemaakt voor GitHub Issues. Iedereen kan hier opmerkingen over de website plaatsen. Dit kunnen bugmeldingen zijn, voorstellen tot veranderen van design, vragen, ... We bekijken allemaal regelmatig de Issuespagina op GitHub om zo up to date te blijven. Er waren de laatste weken zo goed als geen lessen waar we allemaal naartoe gingen, dus was het moeilijk om een moment te vinden om allemaal samen te zitten.\\
\\
Bij de Issues kunnen we makkelijk een taakverdeling bepalen door mensen te assignen aan verschillende Issues.
\\
\\
We hebben Issues aangemaakt voor alles wat nog moet gebeuren qua verplichte functionaliteit. Concreet is dit momenteel: 
\begin{enumerate}
\item Om de zoveel tijd data updaten via een cronjob
\item User groups toevoegen om zo in groepjes tegen elkaar te betten (nu grotendeels klaar)
\item Data op verschillende manieren sorteren en filteren
\item Afwerken van voorspellingsalgoritme
\item Grafisch weergeven van data
\end{enumerate}
Verder hebben we ook een overzicht gemaakt van extra functionaliteit die we gaan implementeren:
\begin{enumerate}
\item Uitbreiden van Facebookintegratie (bv. sharen naar Facebookprofiel)
\item Twitterfeed van teams op teampagina's
\item Naast een nieuwsfeed op de homepage, gaan we ook op teampagina's een feed plaatsen met nieuws specifiek voor het huidige team.
\end{enumerate}
Daarnaast zijn er ook heel wat Issues gemaakt betreffende kleinere bugs, maar de meeste daarvan zijn ondertussen al gefixt.
\\
\\
De grootste updates sinds de laatste evaluatie zijn het updaten van het betsysteem en het toevoegen van usergroups en een notificatiesysteem.
\subsection{Betsysteem}
Betten is nu enkel nog mogelijk via een knop op de pagina's van toekomstige matchen. Voorheen was er ook een betpagina waar de teams en datum handmatig moesten worden ingevuld, maar zoals opgemerkt op de evaluatie is dat niet echt nuttig. Het betformulier verschijnt nu niet meer in een volledig nieuwe pagina, maar wel in een zogenaamde 'modal'. Modals werden ook al gebruikt bij het inloggen. Een modal lijkt wat op een pop-up; het verschijnt bovenop de huidige pagina, maar dan binnen hetzelfde browservenster. Het "first goal" veld is nu een dropdown geworden waarbij de gebruiker kan voorspellen welk van de twee teams het eerste doelpunt scoort. Verder zijn de velden gelijk gebleven. Bij ongeldige input blijft het formulier in een modal geopend, waarbij dan wordt gemeld wat de foute input precies is. Als de bet geaccepteerd wordt, verschijnt er een melding, terug in een modal. Eens deze weggeklikt is, bevindt de gebruiker zich terug op de pagina van de match. Merk op dat meerdere keren betten op dezelfde match momenteel nog mogelijk is, dit moet nog aangepast worden. \\ \\
Aan elke speler wordt een bet score toegekend. Een pas geregistreerde speler begint met een vast aantal punten (nu 1000) en kan er bijverdienen of verliezen afhankelijk van de correctheid van z'n bets. Er werd gewerkt aan het evalueren van de correctheid van een bet. Om het uur wordt er via een cronjob gecheckt of er bets op pas gespeelde matches moeten worden ge\"evalueerd. Het algoritme om de verdiende/verloren punten te bepalen is nog redelijk basic; voor elke correcte gok worden punten verdiend (precieze aantal varieert per item) en voor elke foutieve gok worden punten afgetrokken (niet noodzakelijk evenveel als kan verdiend worden met een correcte gok). Merk op dat niet-ingevulde items als goed noch fout worden gerekend.

\subsection{Notificatiesysteem}
Momenteel ondersteunen we ook notificaties. Wanneer een gebruiker ingelogd is, zal er bovenaan naast het user menu een nieuw item tevoorschijn komen. Wanneer er een notificatie is voor de gebruiker, dan wordt dit hier getoond. Zonder op het item te klikken kan je al zien hoeveel ongelezen notificaties je hebt. Wanneer je erop klikt, zal er een dropdown menu tevoorschijn komen dat de notificaties toont. \\ 
Om dit mogelijk te maken is er een notificatiesysteem geschreven dat gemakkelijk uitbreidbaar is. Elke notificatie bestaat uit een specifiek formaat: er is altijd een actor, een subject, een object en een type. Aan de hand van deze informatie kan er dan een gepaste boodschap worden getoond. 
\begin{enumerate}
\item Actor: De uitvoerende partij
\item Subject: De persoon voor wie de notificatie bestemd is
\item Object: De specifieke informatie over de notificatie
\item Type: Om duidelijk te maken welk voor type notificatie we hebben (Bijvoorbeeld een usergroup invite)
\end{enumerate}
De manier waarop het notificatiesysteem ge\"implementeerd is, zorgt ervoor dat we zonder al te veel moeite een nieuw soort notificatie kunnen toevoegen.

\subsection{Usergroups}
Usergroups zijn nu ook aanwezig in ons systeem: een gebruiker kan een (al dan niet private) usergroup opstarten en mensen inviten. De uitgenodigde user wordt hiervan op de hoogte gebracht door middel van een notificatie, waarna hij de invite kan accepteren of afwijzen.
\subsubsection{Public usergroups}
Elke persoon kan deze usergroup joinen en mensen uitnodigen.
\subsubsection{Private usergroups}
Enkel members in deze groep kunnen mensen uitnodigen. Andere mensen kunnen deze groep wel zien maar kunnen buiten de groepnaam geen details zien. (De link is niet klikbaar en de achtergrond is lichtgrijs)

\subsection{User profiles}
Gebruikers hebben nu ook hun eigen profiel: dit profiel toont wat informatie over de gebruiker zoals in welk land ze wonen, in welke groepen ze zitten, hun betscore, ...
Als je als gebruiker je eigen profiel bekijkt kan je ook je notificaties bekijken en je bio aanpassen.
Er is in het menu nu ook een "Users" link waarin je alle andere gebruikers kan zien (en doorklikken naar hun profiel).

\section{Afspraken}
De deadline komt dichterbij, iedereen moet dus goed doorwerken aan de aan hem geassignde issues.


\section{Planning}
Waarschijnlijk wordt er buiten de momenteel geplande functionaliteit geen extra functionaliteit meer toegevoegd. We moeten dus vooral afwerken waar we nu aan bezig zijn.
\subsection{Stijn}
Ontbrekende data toevoegen via crawler
\subsection{Tom}
Voorspellingsalgortime afwerken
\subsection{Kristof}
User groups, user profiles en notificatiesysteem afwerken
\subsection{Ruben}
Searchfunctie verbeteren
\subsection{Jakob}
Betsysteem afwerken
\subsection{Nog aan iemand toe te wijzen}
Grafische voorstelling van data, ordenen/filteren van data, extra functionaliteit


\end{document}
