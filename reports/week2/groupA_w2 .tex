\documentclass[11pt, a4paper]{article}

\begin{document}
\title{}
\author{Groep A\\ Week 2}
\date{20 februari 2014}
\maketitle

\section{Aanwezigheden}
Iedereen was aanwezig.
\section{Status}
Dit was de eerste echte vergadering. Het project bevindt zich nog in de beginfase.

\section{Besproken onderwerpen}
We hebben een gesprek gehad over welke technologie\"en we gaan gebruiken om onze website te maken. We zijn tot de consensus gekomen om een LAMP systeem te gebruiken. Daaruit volgt dus dat we PHP gekozen ipv Django/Ruby/... Ook is PHP nog altijd de standaard op vlak van webdevelopment.
\newline
We zullen ook een PHP framework gebruiken om onze website te developen, namelijk Laravel. We zullen echter niet alle snufjes van het framework gebruiken. Zo gaan we, zoals opgegeven, nog steeds raw queries  schrijven (het is nog steeds een website voor een databasegerelateerd vak). Ook de User class van Laravel gaan we niet gebruiken. We schrijven ons eigen loginsysteem, wat ook een database gebruikt.
\newline
Qua visuele vormgeving van onze website zullen we gebruik maken van Bootstrap.
\newline
Tijdens de vergadering zijn we ook al begonnen met de setup van Laravel en Boostrap en kregen we op iedereens laptop een lokale server via Apache2 draaiende. We maakten ook een eerste schets van het databaseschema, echter nog zonder foreign keys, triggers, relational integrity, ...
\newline
Verder zijn we nog wat op zoek gegaan naar verschillende voetbaldatabases. We hopen snel een key te krijgen die ons toegang verleent tot de Openfooty API, maar zoniet moeten we een andere bron vinden.
\newline
Ten slotte hebben we als naam voor onze website "Coach Center" gekozen en het domein www.coachcenter.net vastgelegd.

\section{Afspraken}
\begin{enumerate}
\item LAMP (Linux, Apache2, MySQL, PHP5)
\item Laravel 4
\item Twitter Bootstrap
\end{enumerate}


\section{Planning}
We proberen tegen volgende week de basis van de website draaiende te krijgen. Registreren en inloggen moet mogelijk zijn, de databaseschema's afgewerkt, en we hopen de eerste alphaversie van de website online te krijgen.
\subsection{Stijn}
Onderzoekt hoe het bouwen van een basic crawler systeem best aangepakt kan
worden (liefst gebruik makend van een framework), zodat coachcenter gemakelijk
en automatisch gegevens van andere sites kan overnemen.
\subsection{Tom}
Aanmaken van specifieke pagina's voor de website en navigatie.
\subsection{Ruben}
Aanmaken van templates voor pagina's van de websites (header, layout template, ...)
\subsection{Jakob}
Registratie- en inlogsysteem
\subsection{Kristof}
Database schema revisen + foreign dependencies instellen.

\end{document}
