\documentclass[11pt, a4paper]{article}

\begin{document}
\title{}
\author{Groep A}
\date{21 mei 2014}
\maketitle

\section{Aanwezigheden}
Jakob, Kristof, Ruben en Tom waren aanwezig
\section{Status}
Verplichte functionaliteit is bijna af. Enkele dingen moeten nog gedaan worden, maar niets wat erg veel tijd zou mogen kosten.
\subsection{Stijn}

\subsection{Kristof}
Werkte notificatiesysteem af. Voegde emailfunctionaliteit (voor bet reminders) toe.
\subsection{Jakob}
Zocht Twitteraccounts van alle nationale teams. Werkte world map op homepage af en begon aan grafische weergave van data.
\subsection{Ruben}
Herwerkte user profile
\subsection{Tom}
Werkte prediction af. Maakte sorteerbare tabellen.

\section{Besproken onderwerpen}
We besloten een vergadering via Google Hangouts te doen, wat erg goed verliep. We hebben overlopen wat z\'eker nog moet worden gedaan, en wat we nog gaan doen als we de tijd ertoe vinden\\
\\
Grafische voorstelling van data en voorspellingen moet afgewerkt worden, tabellen moeten door de gebruiker gesorteerd en gefilterd kunnen worden. Zowel grafische voorstelling als tabellen sorteren is al ge\"implementeerd, alleen lijkt het nog een probleem om het te tonen binnenin een Ajax tab. Bepaalde stukken Javascript weigeren momenteel samen te werken. Verder moet ook zeker het automatisch updaten van data ge\"implementeerd en getest worden.
\\
\\
\subsection{Grafische weergave}
Voor de grafische weergave van data gebruiken we Google Charts. Dat biedt heel wat verschillende grafieken. De wereldkaart op de homepage maakt deel uit van het aanbod van Google Charts. Momenteel zijn twee grafieken beschikbaar. Er is een taartdiagram dat het percentage gewonnen, verloren en gelijkgespeelde matchen van een team toont en een staafdiagram dat het gemiddelde aantal goals per match per jaar toont. Die tweede wordt nog uitgebreid met gemiddeld aantal kaarten per jaar. Het wordt mogelijk voor de gebruiker om te kiezen of hij goals, kaarten of allebei wil zien. Ook komt er een optie om het totale aantal goals/kaarten in de plaats van het gemiddelde te zien. Indien mogelijk zou de gebruiker ook moeten kunnen het type grafiek aanpassen. De grafieken zullen beschikbaar zijn onder een tab "Graphs" op de Teampagina's.
\\
\\
\section{Afspraken}
We hebben afgesproken de komende dagen elke avond een Hangout te organiseren om de vooruitgang goed op te volgen. Dit doen we tot alle verplichte functionaliteit afgewerkt is.


\section{Planning}
Iedereen moet nog iets van verplichte functionaliteit afwerken
\subsection{Stijn}
Kaarten crawlen. Automatisch updaten van data.
\subsection{Tom}
Sorteren/filteren van tabellen
\subsection{Kristof}

\subsection{Ruben}
Oplossen Ajaxgerelateerde bugs. 
\subsection{Jakob}
Grafische weergave van data afwerken


\end{document}
