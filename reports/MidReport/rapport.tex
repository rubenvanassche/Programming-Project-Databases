\documentclass[11pt, a4paper]{article}

\usepackage{listings}
\usepackage{color}

\definecolor{dkgreen}{rgb}{0,0.6,0}
\definecolor{gray}{rgb}{0.5,0.5,0.5}
\definecolor{mauve}{rgb}{0.58,0,0.82}

\lstset{frame=tb,
  language=SQL,
  aboveskip=3mm,
  belowskip=3mm,
  showstringspaces=false,
  columns=flexible,
  basicstyle={\small\ttfamily},
  numbers=none,
  numberstyle=\tiny\color{gray},
  keywordstyle=\color{blue},
  commentstyle=\color{dkgreen},
  stringstyle=\color{mauve},
  breaklines=true,
  breakatwhitespace=true
  tabsize=3
}

\begin{document}
\title{}
\author{Groep A\\ Rapport 1}
\date{19 maart 2014}
\maketitle


\section{Status}
De website staat momenteel nog niet online, maar de locale versie werkt wel al.	Erg veel features zijn er nog niet, maar verschillende pagina's zijn al beschikbaar. Sommige van deze pagina's gebruiken data uit onze database. Er kunnen ook al accounts aangemaakt worden en inloggen is mogelijk. De features voor ingelogde gebruikers zijn echter nog niet ge\"implementeerd. Merk op dat niet elke knop op de website al werkt!

\section{Taakverdeling}
\subsection{Stijn}
Ging op zoek naar verschillende bronnen voor data. Zocht uit hoe een web crawler nu precies werkt en programmeerde er zelf een specifiek voor onze doeleinden.
\subsection{Kristof}
Werkte het door iedereen samen opgestelde databaseschema af. Maakte heel wat webpagina's aan. Assisteerde Tom bij het voorspellingsalgoritme.
\subsection{Tom}
Werkte aan het voorspellingsalgoritme. Assisteerde Kristof bij het aanmaken van webpagina's.
\subsection{Jakob}
Bouwde het loginsysteem. Bouwde een klasse die het inserten van data in de database encapsuleert.
\subsection{Ruben}
Maakte een template voor de wepagina's. Converteerde iedereens code naar een mooie MVC-structuur om de code flexibeler en makkelijker leesbaar te maken en bracht andere verbeteringen aan in de code.

\section{Design}
De meesten van ons groepje hadden totaal geen ervaring met het bouwen van websites. In het begin van het project zijn de meesten dus begonnen met het doornemen van allerlei tutorials. Dit nam redelijk wat tijd in beslag, waardoor het eigenlijke project met wat vertraging begon. In eerste geval werkten we aan de backend van de website: opstellen van databaseschema, bouwen van loginsysteem, ... We waren eerst van plan om onze data te verkrijgen via de OpenFooty API. Ongeveer een week na het versturen van onze API key request begonnen we te vrezen dat we deze niet meer gingen krijgen, wat inderdaad nog steeds niet gebeurd is. We moesten dus op zoek gaan naar andere manieren om onze data te vergaren. De enige andere optie leek het gebruik van een crawler. Niemand van ons had enige ervaring met het schrijven/gebruiken van crawlers, en de crawler bleek een struikelblok. Ondertussen werkt de crawler min of meer, maar hij is nog niet in een ver genoeg stadium om data te verzamelen die compleet genoeg is om nuttig te zijn. We werken momenteel dus nog met manueel ingevoerde testdata. Ons voorspellingsalgoritme is al uitgedacht maar nog niet geïmplementeerd. Verder zijn al sommige, maar niet alle webpagina's aangemaakt en is er een (tijdelijke?) vormgeving van de pagina's.

\subsection{UML-diagramma}
?

\section{Database}
Onze database bestaat 12 tabellen met voetbalgerelateerde data en 1 tabel voor users.
\begin{enumerate}
\item 'continent': Tabel voor werelddelen. Bevat een id en een naam.
\item `country`: Tabel voor landen. Bevat een id, een naam, de id van het werelddeel waarin het land ligt, en een afkorting voor de naam van het land. Die afkorting zal gebruikt worden op de website.
\item `player`: Tabel voor voetbalspelers. Bevat een id, een naam en een boolean die aangeeft of de speler geblesseerd is.
\item `coach`: Tabel voor voetbalcoaches. Bevat een id en een naam.
\item `team`: Tabel voor voetbalteams. Bevat een id, een naam, een id van het land van het team en een id van de huidige coach van het team.
\item `competition`: Tabel voor voetbalcompetities. Bevat een id en een naam.
\item `match`: Tabel voor voetbalmatches. Bevat id's van thuis- en uitteam, id van de competitie en een datum.
\item `playerPerTeam`: Tabel die spelers en teams met elkaar linkt. Bevat id's van een speler en een team.
\item `playerPerMatch`: Tabel die spelers en matches met elkaar linkt. Bevat id's van een speler en een match en de tijden waarop de speler op het veld kwam en van het veld ging.
\item `teamPerCompetition`: Tabel die teams en competities met elkaar linkt. bevat id's van een team en een competitie.
\item `goal`: Tabel voor doelpunten. Bevat id van match waarin doelpunt gescoord is, tijdstip waarop, id van de speler die het doelpunt scoorde, id van het team waarnaar het punt ging en een boolean die aangeeft of het doelpunt tijdens de penaltyfase gescoord werd.
\item `cards`: Tabel voor gele en rode kaarten. Bevat een id,	een id van de speler die de kaart kreeg, een id van de match waarin de kaart gegeven werd, de kleur van de kaart en de tijd waarop de kaart gegeven werd.
\end{enumerate}

\subsection{ER-diagramma}
\subsection{Constraints}
\begin{lstlisting}
-- Constraints for table `cards`
--
ALTER TABLE `cards`
  ADD CONSTRAINT `match` FOREIGN KEY (`match_id`) REFERENCES `match` (`id`) ON DELETE NO ACTION ON UPDATE CASCADE,
  ADD CONSTRAINT `player` FOREIGN KEY (`player_id`) REFERENCES `player` (`id`) ON DELETE CASCADE ON UPDATE CASCADE;

--
-- Constraints for table `country`
--
ALTER TABLE `country`
  ADD CONSTRAINT `continent` FOREIGN KEY (`continent_id`) REFERENCES `continent` (`id`);

--
-- Constraints for table `goal`
--
ALTER TABLE `goal`
  ADD CONSTRAINT `goal_match` FOREIGN KEY (`match_id`) REFERENCES `match` (`id`) ON DELETE NO ACTION ON UPDATE CASCADE,
  ADD CONSTRAINT `goal_player` FOREIGN KEY (`player_id`) REFERENCES `player` (`id`) ON DELETE NO ACTION ON UPDATE CASCADE,
  ADD CONSTRAINT `goal_team` FOREIGN KEY (`team_id`) REFERENCES `team` (`id`) ON DELETE NO ACTION ON UPDATE CASCADE;

--
-- Constraints for table `match`
--
ALTER TABLE `match`
  ADD CONSTRAINT `awayteam` FOREIGN KEY (`awayteam_id`) REFERENCES `team` (`id`) ON DELETE CASCADE ON UPDATE CASCADE,
  ADD CONSTRAINT `hometeam` FOREIGN KEY (`hometeam_id`) REFERENCES `team` (`id`) ON DELETE CASCADE ON UPDATE CASCADE,
  ADD CONSTRAINT `match_competition` FOREIGN KEY (`competition_id`) REFERENCES `competition` (`id`);

--
-- Constraints for table `playerPerMatch`
--
ALTER TABLE `playerPerMatch`
  ADD CONSTRAINT `player_per_match` FOREIGN KEY (`match_id`) REFERENCES `match` (`id`) ON DELETE CASCADE ON UPDATE CASCADE;

--
-- Constraints for table `playerPerTeam`
--
ALTER TABLE `playerPerTeam`
  ADD CONSTRAINT `player_per_team` FOREIGN KEY (`team_id`) REFERENCES `team` (`id`) ON DELETE CASCADE ON UPDATE CASCADE;

--
-- Constraints for table `team`
--
ALTER TABLE `team`
  ADD CONSTRAINT `team_coach` FOREIGN KEY (`coach_id`) REFERENCES `coach` (`id`) ON DELETE NO ACTION ON UPDATE CASCADE,
  ADD CONSTRAINT `team_country` FOREIGN KEY (`country_id`) REFERENCES `country` (`id`);

--
-- Constraints for table `teamPerCompetition`
--
ALTER TABLE `teamPerCompetition`
  ADD CONSTRAINT `tpc_competition` FOREIGN KEY (`competition_id`) REFERENCES `competition` (`id`);

\end{lstlisting}

\section{User interface}

Hoewel Laravel beschikt over een volledig uitgewerkt loginsysteem, besloten we ons eigen systeem te schrijven. Een dergelijk systeem gebruikt een database voor het opslaan van users, dus is het niet meer dan logisch dat we dit in een vak met betrekking tot databases zelf gaan programmeren. In de database bestaat een tabel 'user', met als fields 'id', 'username', 'firstname', 'lastname', 'email', 'password', 'country', 'session\_id' en 'registrationcode'. De id is een auto incrementing integer die dient als key voor elke entry van de tabel. We voorzien een username, aangezien dit gebruikers een zekere vorm van anonimiteit biedt op onze website. Daarnaast is het een makkelijke manier om gebruikers op de website een unieke benaming te geven. Voornaam en familienaam zijn apart opgeslagen, zodat we in bijvoorbeeld emails gebruikers niet altijd met hun volledige naam niet hoeven aan te spreken. Het paswoord is uiteraard gehasht opgeslagen. Het hashen gebeurt via Laravel, aan de hand van bcrypt. Bcrypt is gebaseerd op het Blowfishalgoritme en heeft een salt ingebouwd, wat accounts beschermt tegen aanvallen gebruik makende van rainbow tables. Laravel biedt een functie voor het vergelijken van een ongehasht en gehasht paswoord. We vragen gebruikers ook in welk land ze wonen, zo kunnen we bijvoorbeeld bepaalde competities en nieuwsberichten een prominentere plek op de website geven afhankelijk van de gebruiker. De session\_id wordt gebruikt om bij te houden of een gebruiker ingelogd is. We plaatsen een tijdelijke cookie met hetzelfde id bij de gebruiker, en kunnen dit zo nagaan. Ten slotte wordt de registratiecode gebruikt bij het nagaan van de validiteit van het emailadres van een gebruiker. De registratiecode wordt gemaild naar de gebruiker, en het account wordt pas geactiveerd wanneer deze code ingegeven wordt. Merk op dat deze functionaliteit momenteel nog niet actief is, gezien we nog niet beschikken over een mailserver.
\\
\\
Voor de validatie van input bij registreren (zijn verplichte velden ingevuld, staat bij emailadres wel een emailadres, is tweemaal hetzelfde paswoord ingetypt, ...) is de validatie van Laravel gebruikt. Dergelijke validatiecode ziet er heel wat beter uit voor de programmeur dan via een hoop simpele if-statements en reguliere expressies. We gebruiken de prepared statements van Laravel om sql-injecties tegen te gaan. Verder gebruiken we de Laravel Query Builder voor makkelijker schrijven van SQL queries niet, dit gebeurt zoals opgegeven met ruwe SQL queries.
\\
\\
Op elke webpagina verschijnt voor een niet-ingelogde gebruiker een loginknop. Het loginmenu wordt dan over de huidige pagina weergegeven. Daar is ook een "forgot password"-functionaliteit voorzien. De gebruiker geeft of zijn username of zijn emailadres in, en er wordt een email gestuurd waarmee het paswoord gereset kan worden. Wederom, deze functionaliteit is nog niet actief. Bij een ingelogde gebruiker wordt bij elke bezochte pagina de login cookie gerefresht, zodat een gebruiker bij langdurige sessies op de site niet om de zoveel tijd opnieuw hoeft in te loggen. Later zal een ingelogde gebruiker ook een knop te zien krijgen die linkt naar zijn persoonlijk controlepaneel, maar dit moet nog ge\"implementeerd worden. Aangezien alle functionaliteit die weggelegd is voor ingelogde gebruikers gebaseerd is op voorspellingen uitbrengen, wat momenteel nog niet ge\"implementeerd is, ziet de website er verder voor

\section{Extra functionaliteit}
We gaan meteen data van zoveel competities als we kunnen vinden in onze database steken en daar hetzelfde mee doen als met de WK-data. Met verdere extra functionaliteit gaan we ons pas bezighouden als de verplichte functionaliteit er volledig is.

\section{Planning}
We proberen zo snel mogelijk alle verplichte functionaliteit af te krijgen. Concreet is dit de database vullen en de data makkelijk beschikbaar maken (in tekst en grafieken), gebruikers voorspellingen laten uitbrengen en het systeem voorspellingen laten maken, data in real time laten updaten en nog wat kleinere dingen.
\subsection{Stijn}
\subsection{Tom}
\subsection{Ruben}
\subsection{Jakob}
\subsection{Kristof}

\end{document}
